\chapter{Requirements}
\label{chapter:requirements}
It was agreed with my supervisor that the project should satisfy the following requirements:

\begin{itemize}
	\item \textbf{Accessibility:} The workbench needs to be accessible to as many users as possible. This requirement is broken down into several sub-requirements:
	
		\begin{itemize}
			\item \textbf{Web based:} The workbench must be implemented in such a way that it is accessible from the Internet, and runs in the user's browser (rather than needing to be downloaded and installed) 
			
			\item \textbf{Plug-in free:} It should run in the browser without requiring third-party plug-ins such as Adobe Flash or Microsoft Silverlight. This effectively limits the implementation language to JavaScript or a language which can be compiled to JavaScript
			
			\item \textbf{Simple UI:} The user interface must be simple to understand and use. Users should be able to use familiar actions such as box selection, deletion using the Delete key, drag-and-drop, etc.
		\end{itemize}
	
	\item \textbf{Non-simplistic simulation:} While the workbench should be easy to use, it should not simply the simulation beyond what is reasonable. The simulation should take in to account the uncertainty in a gate's propagation delay exhibited by real logic gates
\end{itemize}

