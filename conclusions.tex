\chapter{Conclusions}
Evaluating GatePlay against each of the requirements stated in chapter~\ref{chapter:requirements} I reach the following conclusions:

\paragraph{Accessibility}
GatePlay runs in all modern web browsers without the need for plug-ins. The feedback I received from those who saw GatePlay running was that the user interface was appealing and simple to use. Some people found the workflow for creating wires slightly unintuitive (you single click to start drawing, rather than click and hold), so a miniature tutorial to guide new users through the interface would certainly be of benefit.

\paragraph{Simulation}
The simulation is at the right level

\subsection{Future Work}
While I am very pleased with GatePlay as it is today, there are some features that would improve it further.

\subsubsection{Wires over components}
It was always the intention that wires should be able to cross each other, but not cross over components. The workbench Circuit model is able to check if wires cross over any components, but the handler code for moving components became extremely complicated when it had to account for the position of wires. Ultimately I did not resolve the issue, and wires can travel over components.

\subsubsection{Saving circuits as black-boxes}
I useful feature would be the ability to create circuits and then save them for use as a black-box later. GatePlay has an existing notion of inputs ($Toggle$s) and outputs ($LED$s) and it would not be much more work to allow users to ``compile'' their workbench to a black-box.

\subsubsection{Simulation oscilloscope}
The circuit simulator emits a stream of wire value changes as it runs. In GatePlay, these are handled by the Application class and used to colour the wires on the workbench. In addition to this, it would be useful to be able to view traces of the wire values through time (like the one I drew manually in figure~\ref{fig:2-valued-trace}.