\chapter{Testing}
Testing is extremely important in the development of non-trivial programs. Tests give some assurance to the correctness of the code, and highlight \textit{regressions} (where code that used to work is broken by a recent change) quickly.

\section{Unit Testing}
Unit testing is used to test the correctness of small modules of code, such as functions. The simulator has a suite of unit tests. One example is a test over the $AND$ evaluation function shown in figure~\ref{fig:andtest}. All \textit{evaluate} functions had to be changed when $Blinker$s were implemented, and this test flagged a regression when the first implementation had a bug.

\begin{figure}
\begin{lstlisting}[language=JavaScript]
var T = TruthValue.TRUE;
var F = TruthValue.FALSE;
var U = TruthValue.UNKNOWN;

function ANDTest() {
	var and = Functions.get("and");	
	
	// Test 2-input truth table
	equal(and.evaluate([T,T]), T);
	equal(and.evaluate([T,F]), F);
	equal(and.evaluate([T,U]), U);
	equal(and.evaluate([F,T]), F);
	equal(and.evaluate([F,F]), F);
	equal(and.evaluate([F,U]), F);
	equal(and.evaluate([U,T]), U);
	equal(and.evaluate([U,F]), F);
	equal(and.evaluate([U,U]), U);
}
\end{lstlisting}
\caption{Testing $AND$'s truth table}
\label{fig:andtest}
\end{figure}

Unit tests are also very useful for catching simple bugs in frequently edited code. For example, the \textit{tick} function in the circuit simulator is responsible for handling all events which occur at the current time, and them increments the current time. Since the \textit{tick} function is so frequently modified, it is easy to introduce simple bugs. The test shown in figure~ simply checks that the system clock is incremented after \textit{tick} is called.

I used a common JavaScript unit testing framework QUnit\footnote{http://qunit.com} to reduce the boilerplate in writing unit tests.

\section{End-to-end Testing}
Unit testing is a bottom-up approach which checks that the smallest modules work as expected in isolation. End-to-end testing is a top-down approach which